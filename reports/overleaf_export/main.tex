\documentclass[11pt,a4paper]{article}

% Packages
\usepackage[utf8]{inputenc}
\usepackage[T1]{fontenc}
\usepackage{amsmath,amssymb,amsfonts}
\usepackage{graphicx}
\usepackage{booktabs}
\usepackage{hyperref}
\usepackage[margin=1in]{geometry}
\usepackage{float}
\usepackage{xcolor}
\usepackage{subcaption}
\usepackage{enumitem}

% Custom commands
\newcommand{\vix}{\text{VIX}}
\newcommand{\spx}{\text{S\&P 500}}

% Title
\title{\textbf{Volatility Regime Prediction Dataset}\\
\large Data Collection and Quality Assessment Report}
\author{Philipp D. Dubach}
\date{\today}

\begin{document}

\maketitle

\begin{abstract}
This report documents the data collection methodology and quality assessment for a volatility regime prediction research project. We collect and integrate data from multiple sources including CBOE (VIX indices, SKEW index, put/call ratios, and futures term structure), Yahoo Finance (S\&P 500 prices), FRED (macroeconomic and volatility indicators), and \textbf{Alpha Vantage Premium} (SPY options chain with implied volatility surface, Greeks, and volume analytics). The final dataset spans from January 2006 to December 2025, covering 5,046 trading days with \textbf{165 features}. A key addition is the Alpha Vantage SPY options dataset providing 896 weekly observations from 2008--2025 with 26 options-derived features including ATM implied volatility, IV skew at multiple delta levels, term structure slope, and aggregate Greeks. We provide comprehensive descriptive statistics, correlation analysis, and an assessment of data quality suitable for machine learning applications in volatility regime forecasting.
\end{abstract}

\tableofcontents
\clearpage

%------------------------------------------------------------------------------
\section{Introduction}
%------------------------------------------------------------------------------

Volatility regime prediction is a fundamental challenge in quantitative finance with applications in portfolio management, risk management, and derivatives pricing. The VIX index, often called the ``fear gauge,'' captures market expectations of near-term volatility derived from S\&P 500 index options. Understanding and predicting transitions between volatility regimes can provide valuable insights for systematic trading strategies.

This report documents our data collection infrastructure, which is designed with the following objectives:
\begin{enumerate}[noitemsep]
    \item \textbf{Comprehensiveness}: Integrate multiple data sources covering spot volatility, forward-looking volatility (VIX term structure), implied volatility surface (Alpha Vantage options), realized volatility, and macroeconomic conditions.
    \item \textbf{Extensibility}: Implement a modular architecture that allows easy replacement of data sources (e.g., transitioning from free to premium data providers).
    \item \textbf{Reproducibility}: Create a fully automated pipeline that can be re-run to update the dataset.
    \item \textbf{Options Analytics}: Capture the full implied volatility surface structure including skew, term structure, and aggregate Greeks for regime prediction.
\end{enumerate}

%------------------------------------------------------------------------------
\section{Data Sources and Methodology}
%------------------------------------------------------------------------------

\subsection{CBOE Volatility Indices}

The Chicago Board Options Exchange (CBOE) provides historical data for various volatility indices. We collect:

\begin{table}[H]
\centering
\caption{CBOE Volatility Index Series}
\label{tab:cboe_indices}
\begin{tabular}{llc}
\toprule
\textbf{Series} & \textbf{Description} & \textbf{Start Date} \\
\midrule
VIX & 30-day implied volatility & 2006-01-03 \\
VVIX & Volatility of VIX & 2006-03-06 \\
VIX9D & 9-day implied volatility & 2011-01-04 \\
VIX3M & 3-month implied volatility & 2009-09-18 \\
VIX6M & 6-month implied volatility & 2008-01-02 \\
SKEW & Tail risk index & 2006-01-03 \\
\bottomrule
\end{tabular}
\end{table}

Each index provides OHLC (Open, High, Low, Close) values, enabling analysis of intraday volatility dynamics.

\subsection{CBOE SKEW Index}

The CBOE SKEW Index measures perceived tail risk in the S\&P 500 distribution. It is derived from out-of-the-money option prices and reflects the market's expectation of extreme negative returns:

\begin{itemize}[noitemsep]
    \item \textbf{SKEW = 100}: Normal distribution (no perceived tail risk)
    \item \textbf{SKEW $>$ 100}: Elevated left-tail risk (crash protection demand)
    \item \textbf{Historical range}: Typically 100--150, with spikes during stress periods
\end{itemize}

Unlike VIX, which measures the overall level of implied volatility, SKEW captures the \textit{asymmetry} in option pricing---specifically, the premium investors pay for downside protection.

\subsection{Put/Call Ratios}

We collect historical put/call ratio data from CBOE's options volume database (available through October 2019):

\begin{table}[H]
\centering
\caption{CBOE Put/Call Ratio Series}
\label{tab:putcall}
\begin{tabular}{llc}
\toprule
\textbf{Series} & \textbf{Description} & \textbf{Coverage} \\
\midrule
TOTAL\_PC & All CBOE options put/call ratio & 2006-11 to 2019-10 \\
INDEX\_PC & Index options put/call ratio & 2006-11 to 2019-10 \\
EQUITY\_PC & Equity options put/call ratio & 2006-11 to 2019-10 \\
VIX\_PC & VIX options put/call ratio & 2006-02 to 2019-10 \\
\bottomrule
\end{tabular}
\end{table}

Each series includes:
\begin{itemize}[noitemsep]
    \item Call volume, put volume, and total volume
    \item Put/call ratio: $\text{PC} = \text{Put Volume} / \text{Call Volume}$
\end{itemize}

\textbf{Interpretation:}
\begin{itemize}[noitemsep]
    \item \textbf{PC $>$ 1}: More puts traded than calls (bearish sentiment or hedging demand)
    \item \textbf{PC $<$ 1}: More calls traded than puts (bullish sentiment)
    \item \textbf{Extreme readings}: Often used as contrarian indicators
\end{itemize}

\subsection{VIX Futures Term Structure}

We construct the VIX futures term structure by downloading individual contract files from CBOE. For each trading day, we identify the front-month through ninth-month contracts and calculate:

\begin{itemize}[noitemsep]
    \item VX1 to VX9: Settlement prices for each contract month
    \item Term structure slope: $\text{VX2} - \text{VX1}$ and $\text{VX4} - \text{VX1}$
\end{itemize}

The term structure slope is a key indicator of market sentiment:
\begin{itemize}[noitemsep]
    \item \textbf{Contango} (slope $> 0$): Normal market conditions; near-term volatility expected to be lower than future volatility
    \item \textbf{Backwardation} (slope $< 0$): Stressed conditions; elevated near-term volatility expectations
\end{itemize}

\subsection{S\&P 500 Index Data}

We obtain S\&P 500 (ticker: \^{}GSPC) price data from Yahoo Finance via the \texttt{yfinance} library. This provides daily OHLCV data for computing realized volatility measures.

\subsubsection{Realized Volatility}

We compute multiple realized volatility estimators:

\textbf{Standard Realized Volatility:}
\begin{equation}
RV_t^{(n)} = \sqrt{\frac{252}{n} \sum_{i=0}^{n-1} r_{t-i}^2}
\end{equation}
where $r_t = \log(P_t / P_{t-1})$ is the log return.

\textbf{Parkinson Volatility:}
\begin{equation}
\sigma_{P,t}^{(n)} = \sqrt{\frac{1}{4n \log 2} \sum_{i=0}^{n-1} \left(\log\frac{H_{t-i}}{L_{t-i}}\right)^2}
\end{equation}
where $H_t$ and $L_t$ are daily high and low prices. The Parkinson estimator is more efficient than the close-to-close estimator when intraday data is available.

\subsection{Macroeconomic Data (FRED)}

We collect macroeconomic indicators from the Federal Reserve Economic Data (FRED) API:

\begin{table}[H]
\centering
\caption{FRED Economic Series}
\label{tab:fred_series}
\begin{tabular}{llc}
\toprule
\textbf{Series} & \textbf{Description} & \textbf{Frequency} \\
\midrule
\multicolumn{3}{l}{\textit{Interest Rates}} \\
DFF & Federal Funds Effective Rate & Daily \\
DGS1, DGS2, DGS10, DGS30 & Treasury Yields & Daily \\
T10Y2Y & 10Y-2Y Treasury Spread & Daily \\
T10Y3M & 10Y-3M Treasury Spread & Daily \\
TEDRATE & TED Spread (3M LIBOR - T-Bill) & Daily \\
\midrule
\multicolumn{3}{l}{\textit{Credit Spreads}} \\
BAMLH0A0HYM2 & High Yield Corporate Spread & Daily \\
BAMLC0A0CM & Investment Grade Corporate Spread & Daily \\
\midrule
\multicolumn{3}{l}{\textit{Financial Stress Indices}} \\
NFCI & Chicago Fed Financial Conditions Index & Weekly \\
STLFSI4 & St. Louis Fed Financial Stress Index & Weekly \\
\midrule
\multicolumn{3}{l}{\textit{Economic Uncertainty}} \\
USEPUINDXD & Economic Policy Uncertainty Index & Daily \\
\bottomrule
\end{tabular}
\end{table}

The TED spread (TEDRATE) captures credit risk in the banking system---the difference between 3-month LIBOR and the risk-free T-Bill rate. Spikes in the TED spread historically precede market stress (e.g., 2008 financial crisis).

The Economic Policy Uncertainty Index (USEPUINDXD) is a text-based index measuring policy-related economic uncertainty from newspaper coverage, tax code provisions, and economic forecaster disagreement.

\subsection{Alpha Vantage Premium Options Data}

A significant enhancement to our dataset is the integration of \textbf{Alpha Vantage Premium} options data, providing comprehensive implied volatility surface analytics that complement the VIX-based measures. While VIX represents the market's 30-day expected volatility derived from SPX options, our Alpha Vantage integration extracts granular information from the SPY options chain.

\subsubsection{Data Specifications}

\begin{table}[H]
\centering
\caption{Alpha Vantage Options Dataset Summary}
\label{tab:av_summary}
\begin{tabular}{lc}
\toprule
\textbf{Specification} & \textbf{Value} \\
\midrule
Underlying & SPY (S\&P 500 ETF) \\
Date Range & 2008-03-07 to 2025-12-12 \\
Observations & 896 (weekly frequency) \\
Features & 26 \\
Data Quality Score & 92.3\% (24/26 complete) \\
API Subscription & Premium (75 requests/minute) \\
\bottomrule
\end{tabular}
\end{table}

\textbf{Note}: We use SPY rather than SPX options because the Alpha Vantage HISTORICAL\_OPTIONS endpoint provides complete implied volatility and Greeks data for SPY, while SPX options return null values for these critical fields.

\subsubsection{Feature Extraction}

We extract 26 features from each weekly options snapshot:

\textbf{Implied Volatility Measures:}
\begin{itemize}[noitemsep]
    \item \texttt{AV\_ATM\_IV}: At-the-money implied volatility (strike nearest current price)
    \item \texttt{AV\_CALL\_IV\_MEAN/MEDIAN}: Call options average implied volatility
    \item \texttt{AV\_PUT\_IV\_MEAN/MEDIAN}: Put options average implied volatility
    \item \texttt{AV\_VW\_IV}: Volume-weighted implied volatility across all strikes
\end{itemize}

\textbf{IV Skew Features:}
\begin{itemize}[noitemsep]
    \item \texttt{AV\_IV\_SKEW\_25D}: 25-delta put IV minus 25-delta call IV
    \item \texttt{AV\_IV\_SKEW\_10D}: 10-delta put IV minus 10-delta call IV (deeper OTM)
\end{itemize}

The volatility skew measures the relative expensiveness of out-of-the-money puts versus calls. A positive skew indicates investors pay a premium for downside protection---a hallmark of equity markets since the 1987 crash.

\textbf{IV Term Structure:}
\begin{itemize}[noitemsep]
    \item \texttt{AV\_IV\_TERM\_NEAR}: ATM IV for nearest-dated expiration
    \item \texttt{AV\_IV\_TERM\_FAR}: ATM IV for further-dated expiration
    \item \texttt{AV\_IV\_TERM\_SLOPE}: Far IV minus Near IV (term structure slope)
\end{itemize}

Negative term slope (backwardation) indicates elevated near-term volatility expectations, typically during market stress.

\textbf{Volume and Sentiment:}
\begin{itemize}[noitemsep]
    \item \texttt{AV\_PUT\_CALL\_RATIO\_VOL}: Put/Call volume ratio
    \item \texttt{AV\_PUT\_CALL\_RATIO\_OI}: Put/Call open interest ratio
    \item \texttt{AV\_CALL\_VOLUME}, \texttt{AV\_PUT\_VOLUME}: Absolute volumes
    \item \texttt{AV\_CALL\_OI}, \texttt{AV\_PUT\_OI}: Open interest levels
\end{itemize}

\textbf{Greeks Aggregates:}
\begin{itemize}[noitemsep]
    \item \texttt{AV\_NET\_DELTA}: Net delta exposure (calls minus puts)
    \item \texttt{AV\_TOTAL\_GAMMA}: Sum of absolute gamma values
    \item \texttt{AV\_TOTAL\_VEGA}: Sum of absolute vega values
\end{itemize}

\subsubsection{Statistical Summary}

\begin{table}[H]
\centering
\caption{Alpha Vantage Options Statistics (896 Weekly Observations)}
\label{tab:av_stats}
\begin{tabular}{lccccc}
\toprule
\textbf{Feature} & \textbf{Mean} & \textbf{Std} & \textbf{Min} & \textbf{Max} & \textbf{Skew} \\
\midrule
\multicolumn{6}{l}{\textit{Implied Volatility (\%)}} \\
ATM IV & 16.78 & 7.13 & 1.49 & 59.05 & 0.41 \\
IV Skew 25D & 7.16 & 7.71 & -35.77 & 31.87 & -0.51 \\
IV Skew 10D & 11.19 & 10.26 & -46.50 & 46.83 & -0.42 \\
\midrule
\multicolumn{6}{l}{\textit{Term Structure}} \\
Term Slope & -11.41 & 20.71 & -82.44 & 67.68 & 0.22 \\
\midrule
\multicolumn{6}{l}{\textit{Sentiment}} \\
P/C Ratio (Vol) & 1.52 & 0.34 & 0.46 & 2.93 & 0.55 \\
P/C Ratio (OI) & 1.99 & 0.36 & 1.04 & 3.35 & 0.48 \\
\midrule
\multicolumn{6}{l}{\textit{Greeks}} \\
Net Delta & 0.004 & 0.105 & -0.487 & 0.199 & -1.22 \\
Total Gamma & 46.12 & 15.00 & 13.43 & 84.78 & 0.35 \\
Total Vega & 2090 & 1980 & 21.3 & 11,584 & 1.84 \\
\bottomrule
\end{tabular}
\end{table}

\textbf{Key Observations:}
\begin{enumerate}[noitemsep]
    \item \textbf{ATM IV}: Mean of 16.78\% with maximum of 59.05\% during COVID-19 crash (March 2020). Right-skewed distribution consistent with volatility behavior.
    \item \textbf{IV Skew}: Positive 91.7\% of observations, confirming persistent demand for downside protection. Mean 25D skew of 7.16\% indicates puts trade roughly 7 vol points higher than equivalent calls.
    \item \textbf{Term Structure}: Market in backwardation 51.3\% of observations, contango 2.9\%, flat 45.8\%. Mean slope of -11.4\% indicates elevated near-term concerns are common.
    \item \textbf{Put/Call Ratios}: Volume ratio $>$ 1 for 97.1\% of observations, indicating structural put-buying in SPY options (hedging demand).
    \item \textbf{Net Delta}: Near-zero mean (0.004) suggests balanced positioning, but range [-0.49, 0.20] indicates significant directional shifts.
\end{enumerate}

\subsubsection{Historical Event Validation}

We validate our options data against known market events:

\begin{table}[htbp]
\centering
\caption{Options Data Event Validation}
\label{tab:av_events}
\begin{tabular}{lccc}
\toprule
\textbf{Event} & \textbf{Date} & \textbf{ATM IV} & \textbf{Expected} \\
\midrule
COVID-19 Crash & 2020-03-27 & 59.05\% & $>$ 50\% \\
2008 Financial Crisis & 2008-11-21 & 28.76\% & $>$ 25\% \\
2022 Bear Market & 2022-06-17 & 29.13\% & $>$ 25\% \\
Pre-COVID Calm & 2020-01-03 & 12.68\% & $<$ 15\% \\
2017 Low Vol & 2017-11-03 & 9.61\% & $<$ 12\% \\
\bottomrule
\end{tabular}
\end{table}

All validation checks pass, confirming data integrity.

%------------------------------------------------------------------------------
\section{Dataset Overview}
%------------------------------------------------------------------------------

\subsection{Summary Statistics}

\begin{table}[htbp]
\centering
\caption{Dataset Summary Statistics}
\label{tab:summary}
\begin{tabular}{lc}
\toprule
\textbf{Metric} & \textbf{Value} \\
\midrule
Date Range & 2006-01-03 to 2025-12-12 \\
Trading Days & 5,046 \\
Total Features & \textbf{165} \\
\quad CBOE Volatility Indices & 24 \\
\quad VIX Futures Term Structure & 18 \\
\quad S\&P 500 Price/Returns & 12 \\
\quad FRED Macro Indicators & 24 \\
\quad Derived Features & 61 \\
\quad \textbf{Alpha Vantage Options} & \textbf{26} \\
\midrule
\multicolumn{2}{l}{\textit{VIX Statistics}} \\
Mean & 19.46 \\
Standard Deviation & 8.73 \\
Median & 17.10 \\
Minimum & 9.14 \\
Maximum & 82.69 \\
Skewness & 2.50 \\
Kurtosis & 9.36 \\
\midrule
\multicolumn{2}{l}{\textit{Alpha Vantage ATM IV Statistics}} \\
Mean & 16.78\% \\
Standard Deviation & 7.13\% \\
Median & 17.10\% \\
Minimum & 1.49\% \\
Maximum & 59.05\% (COVID-19) \\
Observations & 896 (weekly) \\
\midrule
\multicolumn{2}{l}{\textit{SKEW Statistics}} \\
Mean & 121.8 \\
Typical Range & 110--135 \\
\midrule
\multicolumn{2}{l}{\textit{Put/Call Ratios (Alpha Vantage)}} \\
Volume PC Mean & 1.52 \\
Open Interest PC Mean & 1.99 \\
\midrule
\multicolumn{2}{l}{\textit{S\&P 500 Returns}} \\
Annualized Mean Return & 10.30\% \\
Annualized Volatility & 19.45\% \\
Sharpe Ratio & 0.53 \\
Worst Daily Return & -11.98\% \\
Best Daily Return & 11.58\% \\
\midrule
\multicolumn{2}{l}{\textit{Options IV Skew (Alpha Vantage)}} \\
25-Delta Skew Mean & 7.16\% \\
10-Delta Skew Mean & 11.19\% \\
Positive Skew Frequency & 91.7\% \\
\midrule
\multicolumn{2}{l}{\textit{Term Structure}} \\
Contango Frequency (VX1 $>$ VIX) & 76.5\% \\
Backwardation (Options) & 51.3\% \\
Upward Slope (VX2 $>$ VX1) & 81.5\% \\
Mean Slope (VX2-VX1) & 0.88 \\
\bottomrule
\end{tabular}
\end{table}

\subsection{VIX Distribution Characteristics}

The VIX exhibits several well-documented statistical properties:

\begin{enumerate}[noitemsep]
    \item \textbf{Right-skewness} (skew = 2.50): Volatility spikes are more common than volatility crashes
    \item \textbf{Excess kurtosis} (kurtosis = 9.36): Fat tails indicate frequent extreme values
    \item \textbf{Mean reversion}: VIX tends to revert to its long-term mean around 19-20
    \item \textbf{Regime-dependent behavior}: Distinct low, normal, and high volatility states
\end{enumerate}

These characteristics motivate regime-switching models rather than simple linear forecasting approaches.

%------------------------------------------------------------------------------
\section{Feature Engineering}
%------------------------------------------------------------------------------

We compute the following derived features for use in predictive modeling:

\subsection{Realized Volatility Features}
\begin{itemize}[noitemsep]
    \item Rolling realized volatility: 5, 10, 21, 63, 126, 252-day windows
    \item Parkinson volatility: 5, 10, 21, 63, 126, 252-day windows
    \item Log returns and log realized volatility
\end{itemize}

\subsection{Variance Risk Premium (VRP)}

The Variance Risk Premium captures the difference between implied and realized volatility:
\begin{equation}
\text{VRP} = \sigma_{\text{IV}}^2 - \sigma_{\text{RV}}^2
\end{equation}

We compute both forward-looking VRP (comparing to future realized volatility, ex-post) and backward-looking VRP (comparing to historical realized, available in real-time). 

\textbf{Important:} Forward VRP (\texttt{vrp\_forward}) uses future realized volatility and \textbf{cannot} be used as a predictor in ML models. It is computed for ex-post analysis only. Use \texttt{vrp\_backward} for real-time available features.

Key VRP statistics in our dataset:
\begin{itemize}[noitemsep]
    \item VRP backward is positive 84.1\% of the time
    \item VRP forward is positive 82.0\% of the time  
    \item Mean VRP: $\sim$3.4 volatility points
    \item VRP turns sharply negative during volatility shocks
\end{itemize}

\subsection{Term Structure Features}
\begin{itemize}[noitemsep]
    \item VIX basis: $\text{VX1} - \text{VIX}$
    \item Term slope: $\text{VX2} - \text{VX1}$, $\text{VX4} - \text{VX1}$
    \item Percentage slopes for standardization
\end{itemize}

\subsection{Volatility of Volatility}
\begin{itemize}[noitemsep]
    \item VVIX index level
    \item Rolling VIX return volatility: 5, 10, 21-day windows
    \item VIX change and range over multiple horizons
\end{itemize}

\subsection{Regime Indicators}
\begin{itemize}[noitemsep]
    \item VIX regime: 0 (low), 1 (normal), 2 (high) based on thresholds
    \item VIX percentile: Expanding window percentile rank (252-day minimum) -- no look-ahead bias
    \item VIX z-score: Rolling 252-day standardized value
    \item Contango indicator: Binary flag for positive term structure slope (76.5\% when VX1 data available)
\end{itemize}

\subsection{SKEW Features}
\begin{itemize}[noitemsep]
    \item SKEW z-score: Rolling 252-day standardized SKEW value
    \item SKEW percentile: Rolling 252-day percentile rank
    \item SKEW changes: 5-day and 21-day changes
    \item SKEW/VIX ratio: Relative tail risk vs. overall volatility
    \item High SKEW regime: Binary indicator for elevated tail risk (SKEW $>$ 130)
\end{itemize}

\subsection{Put/Call Sentiment Features}
\begin{itemize}[noitemsep]
    \item Moving averages: 5-day and 21-day smoothed P/C ratios
    \item P/C z-score: Rolling 63-day standardized value
    \item Extreme P/C indicators: Binary flags for ratios $>$ 1.2 (bearish) or $<$ 0.6 (bullish)
    \item P/C spread: Difference between equity and index P/C ratios
\end{itemize}

\subsection{Alpha Vantage Options Features}

The Alpha Vantage Premium integration provides 26 additional features capturing the full options surface:

\textbf{Implied Volatility Surface:}
\begin{itemize}[noitemsep]
    \item ATM IV, Call IV Mean/Median, Put IV Mean/Median, Volume-weighted IV
    \item IV Skew at 25-delta and 10-delta levels
    \item Near-term and far-term ATM IV, term structure slope
\end{itemize}

\textbf{Volume Analytics:}
\begin{itemize}[noitemsep]
    \item Call/Put volumes and open interest
    \item Put/Call ratios (volume and OI-based)
    \item Total volume and open interest
    \item Contract count per snapshot
\end{itemize}

\textbf{Greeks Aggregates:}
\begin{itemize}[noitemsep]
    \item Net Delta: Directional exposure (calls minus puts)
    \item Total Gamma: Sum of gamma across strikes (convexity exposure)
    \item Total Vega: Sum of vega across strikes (volatility sensitivity)
\end{itemize}

These features complement the VIX-based indicators by providing:
\begin{enumerate}[noitemsep]
    \item \textbf{Higher granularity}: Delta-level IV skew vs. aggregate VIX
    \item \textbf{Different underlying}: SPY options vs. SPX options for VIX
    \item \textbf{Volume/positioning data}: Actual market activity vs. price-based indices
    \item \textbf{Greeks}: Risk sensitivities unavailable from VIX alone
\end{enumerate}

%------------------------------------------------------------------------------
\section{Data Quality Assessment}
%------------------------------------------------------------------------------

\subsection{Missing Data Analysis}

\begin{table}[H]
\centering
\caption{Missing Data Summary (Trading Days Only)}
\label{tab:missing}
\begin{tabular}{lcc}
\toprule
\textbf{Series} & \textbf{Missing \%} & \textbf{Reason} \\
\midrule
VIX (OHLC) & 0\% & Full coverage (trading day filter) \\
S\&P 500 & 0\% & Full coverage \\
VVIX & 4.5\% & Later start date (2006-03-26) \\
VIX9D & 25.8\% & Series started 2011 \\
VIX3M & 20.0\% & Series started 2009 \\
VIX6M & 11.0\% & Series started 2008 \\
SKEW & 2.8\% & Small gaps in early data \\
CBOE Put/Call Ratios & 27\% & Data ends October 2019 \\
VX1--VX5 & 31\% & Futures started 2007 \\
NFCI, STLFSI4 & 68--75\% & Weekly frequency \\
\textbf{Alpha Vantage Options} & \textbf{82\%} & Weekly freq., starts 2008 \\
\quad AV\_OTM\_VOL\_RATIO & 100\% & API limitation (null values) \\
\quad AV\_UNDERLYING & 100\% & API limitation (null values) \\
\bottomrule
\end{tabular}
\end{table}

Missing data arises primarily from:
\begin{enumerate}[noitemsep]
    \item \textbf{Staggered series inception}: Newer indices (VIX9D, VIX3M) have shorter histories
    \item \textbf{Frequency mismatch}: Weekly series (NFCI, Alpha Vantage) mapped to daily trading dates
    \item \textbf{Data availability}: VIX futures electronic trading began 2007
    \item \textbf{Historical data cutoff}: CBOE put/call ratio data ends October 2019
    \item \textbf{API limitations}: Alpha Vantage returns null for some fields (OTM ratio, underlying)
\end{enumerate}

\textbf{Handling Strategy:} For modeling, we recommend either:
\begin{itemize}[noitemsep]
    \item Using a subset of features with complete data (VIX, S\&P 500, Treasury yields)
    \item Restricting the analysis to 2012--2019 when futures and P/C data overlap
    \item Applying forward-fill for weekly economic indicators and Alpha Vantage options
    \item Using separate models for post-2019 with Alpha Vantage options data
    \item Weekly models that align with Alpha Vantage snapshot frequency
\end{itemize}

\subsection{Data Validation}

We implement automated validation checks:
\begin{itemize}[noitemsep]
    \item Non-negative VIX values (all indices should be positive)
    \item Reasonable price ranges (VIX $< 200$, S\&P 500 $> 0$)
    \item Monotonic date indices without gaps
    \item Consistent cross-series relationships (VX1 $\approx$ VIX)
\end{itemize}

%------------------------------------------------------------------------------
\section{Volatility Regime Analysis}
%------------------------------------------------------------------------------

\subsection{Regime Classification}

We define three volatility regimes based on VIX levels:
\begin{itemize}[noitemsep]
    \item \textbf{Low Volatility} (VIX $< 15$): Calm market conditions
    \item \textbf{Normal Volatility} ($15 \leq$ VIX $< 25$): Typical trading environment
    \item \textbf{High Volatility} (VIX $\geq 25$): Elevated uncertainty/stress
\end{itemize}

\subsection{Regime Statistics}

\begin{table}[H]
\centering
\caption{Volatility Regime Characteristics}
\label{tab:regimes}
\begin{tabular}{lccc}
\toprule
\textbf{Regime} & \textbf{Frequency} & \textbf{Avg. VIX} & \textbf{Avg. SPX Return} \\
\midrule
Low Vol (VIX $< 15$) & 30.2\% & 12.5 & Positive \\
Normal Vol ($15 \leq$ VIX $< 25$) & 50.1\% & 18.9 & Near zero \\
High Vol (VIX $\geq 25$) & 19.7\% & 34.2 & Negative \\
\bottomrule
\end{tabular}
\end{table}

Key observations:
\begin{itemize}[noitemsep]
    \item The market spends roughly half of the time in ``normal'' volatility conditions
    \item High volatility regimes, while less frequent, are associated with significant market stress
    \item Regime persistence is high, suggesting Markov-switching models are appropriate
\end{itemize}

%------------------------------------------------------------------------------
\section{Correlation Structure}
%------------------------------------------------------------------------------

\subsection{Primary Correlations}

Key correlations in the dataset:
\begin{itemize}[noitemsep]
    \item \textbf{VIX vs S\&P 500 returns}: Strong negative correlation ($\rho \approx -0.72$ using \% changes, $\rho \approx -0.81$ using point changes)
    \item \textbf{VIX vs VVIX}: Positive correlation ($\rho \approx 0.65$)
    \item \textbf{VIX vs High Yield spread}: Positive correlation ($\rho \approx 0.75$)
    \item \textbf{VIX vs SKEW}: Weak positive correlation; SKEW captures different information
    \item \textbf{VIX term structure}: VX1-VX9 highly correlated but with decreasing magnitude
    \item \textbf{VIX vs Treasury curve}: Weak negative correlation with slope (flattening associated with higher VIX)
    \item \textbf{Put/Call ratios vs VIX}: Moderate positive correlation (higher P/C with elevated VIX)
\end{itemize}

\subsection{Alpha Vantage Options Correlations}

The options surface features reveal additional correlation structure:
\begin{itemize}[noitemsep]
    \item \textbf{ATM IV vs IV Skew (25D)}: Moderate positive correlation ($\rho = 0.243$)
    \item \textbf{ATM IV vs Term Slope}: Weak positive correlation ($\rho = 0.052$)
    \item \textbf{ATM IV vs P/C Ratio}: Weak positive correlation ($\rho = 0.050$)
    \item \textbf{25D Skew vs 10D Skew}: Strong positive correlation ($\rho = 0.91$)
    \item \textbf{Near IV vs Far IV}: Strong positive correlation ($\rho = 0.95$)
    \item \textbf{Call IV vs Put IV}: Strong positive correlation ($\rho = 0.98$), but skew persists
    \item \textbf{Total Gamma vs Total Vega}: Moderate positive correlation ($\rho = 0.58$)
\end{itemize}

\textbf{Implications for Feature Selection:}
\begin{enumerate}[noitemsep]
    \item IV skew provides largely independent information from ATM IV level
    \item Term structure slope is weakly correlated with IV level, offering orthogonal signal
    \item 25D and 10D skews are highly correlated; consider using one or a ratio
    \item Greeks aggregates capture different information than IV-based features
\end{enumerate}

%------------------------------------------------------------------------------
\section{Technical Implementation}
%------------------------------------------------------------------------------

\subsection{Architecture}

The data pipeline implements a modular architecture:

\begin{verbatim}
src/
+-- data/
|   +-- base.py           # Abstract base class for data sources
|   +-- fred.py           # FRED API integration
|   +-- cboe.py           # CBOE web crawler
|   +-- yfinance_source.py    # Yahoo Finance wrapper
|   +-- alpha_vantage.py  # Alpha Vantage Premium API (NEW)
|   +-- data_manager.py   # Orchestration and merging
+-- features/
|   +-- volatility.py   # Feature engineering
+-- analysis/
    +-- eda.py          # Exploratory analysis
\end{verbatim}

\subsection{Extensibility}

The abstract \texttt{BaseDataSource} class provides:
\begin{itemize}[noitemsep]
    \item Standardized interface for all data sources
    \item Built-in caching with configurable expiration
    \item Validation framework
    \item Easy replacement of free sources with premium alternatives
\end{itemize}

\subsection{Reproducibility}

The pipeline can be run with:
\begin{verbatim}
python src/main.py
\end{verbatim}

This will:
\begin{enumerate}[noitemsep]
    \item Download fresh data from all sources
    \item Compute derived features
    \item Save processed dataset to \texttt{data/processed/}
    \item Generate data quality report
\end{enumerate}

%------------------------------------------------------------------------------
\section{Conclusions}
%------------------------------------------------------------------------------

We have constructed a comprehensive volatility dataset suitable for regime prediction research. Key strengths:

\begin{enumerate}[noitemsep]
    \item \textbf{Comprehensive coverage}: Multiple volatility measures, SKEW tail risk, put/call sentiment, term structure data, macro indicators, \textbf{and full options surface analytics}
    \item \textbf{Long history}: Nearly 20 years of data covering multiple market cycles (2006--2025)
    \item \textbf{Research-ready features}: Pre-computed realized volatility, regime indicators, sentiment metrics, term structure features, \textbf{and 26 options-derived features}
    \item \textbf{Multi-dimensional sentiment}: VIX (level), SKEW (tail risk), P/C ratios (positioning), \textbf{and IV skew (options asymmetry)} capture different aspects of market fear
    \item \textbf{Premium data integration}: Alpha Vantage Premium provides granular options surface data unavailable from free sources
    \item \textbf{Extensible design}: Modular architecture allows easy addition of new data sources
\end{enumerate}

\subsection{Data Limitations}

Important limitations to consider:
\begin{itemize}[noitemsep]
    \item \textbf{Forward-looking features}: \texttt{vrp\_forward} and \texttt{vrp\_vol\_points\_forward} use future data and cannot be used as predictors
    \item CBOE put/call ratio data ends October 2019 (discontinued free distribution)
    \item Weekly economic indicators (NFCI, STLFSI) create alignment challenges
    \item VIX futures term structure data has shorter history than spot VIX (VX1 from 2013)
    \item Look-ahead bias must be carefully managed in feature engineering
    \item \textbf{Alpha Vantage limitations}: Weekly frequency limits intraweek analysis; 2 features return null (OTM vol ratio, underlying price)
    \item \textbf{SPY vs SPX}: Alpha Vantage provides SPY options; potential basis vs. SPX-derived VIX
\end{itemize}

\subsection{Data Quality Summary}

\begin{table}[H]
\centering
\caption{Data Quality Scorecard}
\label{tab:quality}
\begin{tabular}{lccc}
\toprule
\textbf{Source} & \textbf{Completeness} & \textbf{Validation} & \textbf{Quality Score} \\
\midrule
CBOE Volatility Indices & 100\% & Passed & A+ \\
Yahoo Finance (S\&P 500) & 100\% & Passed & A+ \\
FRED Economic Data & 95\% & Passed & A \\
VIX Futures & 69\% & Passed & B+ \\
CBOE Put/Call Ratios & 73\% & Passed & B \\
\textbf{Alpha Vantage Options} & \textbf{92.3\%} & \textbf{Passed} & \textbf{A} \\
\bottomrule
\end{tabular}
\end{table}

\subsection{Next Steps}

Recommended modeling approaches for this dataset:
\begin{itemize}[noitemsep]
    \item Hidden Markov Models for regime detection
    \item Causal discovery methods (e.g., PC algorithm, NOTEARS) to identify lead-lag relationships
    \item Machine learning models (Random Forest, XGBoost) for regime prediction
    \item Neural networks for complex nonlinear relationships
    \item \textbf{Weekly models leveraging full Alpha Vantage options surface}
    \item \textbf{IV skew and term structure as regime predictors}
\end{itemize}

\subsection{Potential Improvements}

Future enhancements to consider:
\begin{itemize}[noitemsep]
    \item \textbf{Higher frequency options data}: Daily options snapshots would improve temporal resolution
    \item \textbf{Full Greeks chain}: Individual option-level Greeks rather than aggregates
    \item \textbf{Cross-asset options}: VIX options, other ETF options surfaces
    \item \textbf{Real-time streaming}: Live data integration for production systems
    \item \textbf{Alternative IV sources}: Compare Alpha Vantage IV to other vendors (ORATS, IVolatility)
\end{itemize}

%------------------------------------------------------------------------------
% References
%------------------------------------------------------------------------------

\begin{thebibliography}{9}

\bibitem{cboe_vix}
CBOE (2024). \textit{VIX Index Methodology}. Chicago Board Options Exchange.

\bibitem{fred}
Federal Reserve Bank of St. Louis (2024). \textit{FRED Economic Data}. \url{https://fred.stlouisfed.org/}

\bibitem{parkinson}
Parkinson, M. (1980). The Extreme Value Method for Estimating the Variance of the Rate of Return. \textit{Journal of Business}, 53(1), 61-65.

\bibitem{whaley}
Whaley, R. E. (2009). Understanding the VIX. \textit{Journal of Portfolio Management}, 35(3), 98-105.

\end{thebibliography}

%------------------------------------------------------------------------------
% Appendix: Figures
%------------------------------------------------------------------------------
\appendix
\section{Data Visualizations}

\subsection{VIX and Market Data}

\begin{figure}[H]
    \centering
    \includegraphics[width=\textwidth]{figures/vix_history.png}
    \caption{VIX Index historical time series (2006--2025), showing major volatility spikes during financial crises.}
    \label{fig:vix_history}
\end{figure}

\begin{figure}[H]
    \centering
    \includegraphics[width=\textwidth]{figures/vix_distribution.png}
    \caption{VIX distribution analysis showing the characteristic right-skewed distribution with mean $\approx$ 19.5 and positive skewness.}
    \label{fig:vix_dist}
\end{figure}

\begin{figure}[H]
    \centering
    \includegraphics[width=\textwidth]{figures/spx_returns.png}
    \caption{S\&P 500 daily returns distribution showing fat tails and volatility clustering typical of financial returns.}
    \label{fig:spx_returns}
\end{figure}

\begin{figure}[H]
    \centering
    \includegraphics[width=\textwidth]{figures/futures_term_structure.png}
    \caption{VIX futures term structure dynamics. The market is typically in contango (upward sloping) approximately 76.5\% of the time (when VX1 data available, 2013--2025).}
    \label{fig:term_structure}
\end{figure}

\begin{figure}[H]
    \centering
    \includegraphics[width=\textwidth]{figures/regime_analysis.png}
    \caption{Volatility regime analysis showing the distribution of low, normal, and high volatility regimes over the sample period.}
    \label{fig:regimes}
\end{figure}

\begin{figure}[H]
    \centering
    \includegraphics[width=\textwidth]{figures/correlation_matrix.png}
    \caption{Correlation matrix of key features showing the strong negative VIX-SPX relationship ($\rho \approx -0.81$) and term structure correlations.}
    \label{fig:correlation}
\end{figure}

\begin{figure}[H]
    \centering
    \includegraphics[width=\textwidth]{figures/economic_variables.png}
    \caption{Economic indicators time series including Treasury yields, credit spreads, and financial conditions indices.}
    \label{fig:economic}
\end{figure}

\subsection{Alpha Vantage Options Analysis}

\begin{figure}[H]
    \centering
    \includegraphics[width=\textwidth]{figures/av_atm_iv_timeseries.png}
    \caption{SPY Options ATM Implied Volatility (2008--2025). Mean IV of 16.78\% with maximum of 59.05\% during COVID-19 crash (March 2020). Dashed line indicates long-term mean.}
    \label{fig:av_atm_iv}
\end{figure}

\begin{figure}[H]
    \centering
    \includegraphics[width=\textwidth]{figures/av_iv_distribution.png}
    \caption{ATM IV distribution and Q-Q plot. The distribution exhibits right-skewness (0.41) and excess kurtosis (2.73) consistent with volatility's mean-reverting, fat-tailed nature.}
    \label{fig:av_iv_dist}
\end{figure}

\begin{figure}[H]
    \centering
    \includegraphics[width=\textwidth]{figures/av_iv_skew_analysis.png}
    \caption{IV Skew Analysis. Top-left: 25-delta skew time series. Top-right: Skew distribution (mean 7.16\%). Bottom-left: ATM IV vs skew scatter ($\rho = 0.243$). Bottom-right: 25D vs 10D skew relationship ($\rho = 0.91$).}
    \label{fig:av_skew}
\end{figure}

\begin{figure}[H]
    \centering
    \includegraphics[width=\textwidth]{figures/av_term_structure.png}
    \caption{IV Term Structure Analysis. Top-left: Term slope time series showing frequent backwardation (red). Top-right: Slope distribution (mean -11.4\%). Bottom panels: Relationship between term slope and ATM IV/skew.}
    \label{fig:av_term}
\end{figure}

\begin{figure}[H]
    \centering
    \includegraphics[width=\textwidth]{figures/av_greeks_analysis.png}
    \caption{Aggregate Greeks Analysis. Net Delta (top) oscillates around zero indicating balanced positioning. Total Gamma (bottom-left) and Total Vega (bottom-right) show consistent time evolution.}
    \label{fig:av_greeks}
\end{figure}

\begin{figure}[H]
    \centering
    \includegraphics[width=\textwidth]{figures/av_correlation_matrix.png}
    \caption{Alpha Vantage Options Features Correlation Matrix. Strong correlations between IV measures (Near/Far IV, Call/Put IV) and weak correlations between IV level and sentiment/positioning features.}
    \label{fig:av_corr}
\end{figure}

\begin{figure}[H]
    \centering
    \includegraphics[width=\textwidth]{figures/av_put_call_ratio.png}
    \caption{Put/Call Ratio Analysis. Volume-based (mean 1.52) and OI-based (mean 1.99) ratios consistently above 1, indicating structural put-buying for hedging. Distribution shows OI ratio is higher and more stable.}
    \label{fig:av_pcr}
\end{figure}

\begin{figure}[H]
    \centering
    \includegraphics[width=\textwidth]{figures/av_regime_analysis.png}
    \caption{Options-based Volatility Regime Analysis. Top: ATM IV time series with regime classification (Low $<$ 15\%, Normal 15-25\%, High $>$ 25\%). Bottom: Regime duration distribution showing most regimes last under 100 days.}
    \label{fig:av_regime}
\end{figure}

\end{document}
